\documentclass{article}
\usepackage{natbib, palatino, fullpage, hyperref}
\title{Probabilistic Programming Reading List}
\author{Steven Holtzen\\\url{s.holtzen@northeastern.edu}}
\begin{document}
\maketitle

\begin{abstract}
  This document contains the Fall 2021 reading list for CS7480: Topics in
  Probabilistic Programming. The goal of this document is to be a representative
  list of papers from which students can draw from for presentations. It is not
  exhaustive, but if anyone would like to add an article, please feel free to
  make a pull request at the GitHub repository 
\end{abstract}

\tableofcontents

\section{Semantics}
\begin{enumerate}
\item \emph{Semantics of probabilistic programs}, \citep{Kozen1981}: One of the
  original semantics papers on probabilistic programs, designed for verifying and
  representing randomized algorithms.
\item \emph{PMAF: an algebraic framework for static analysis of probabilistic
    programs}, \citep{Wang2018}: Gives an algebraic semantics for manipulating
  probabilistic programs; very recent.
\item \citep{Borgstrom2013}: Measure transformer semantics
\item \citep{Scibior2017}: Describe a semantics of recursive probabilistic
  programs, discrete inference
\item \citep{Vakar2019}: Domain theory for higher-order probabilistic programs
\end{enumerate}

\section{Probabilistic Program Analysis}

\item \citep{Morgan1996}: Probabilistic predicate transformers
\item \citep{FerrerFioriti2015}: Analysis of probabilistic termination.


\section{Inference \& Systems}
In general, inference methods can be divided into several broad categories:
\begin{enumerate}
\item \emph{Approximation} methods, which perform approximate inference either
  via sampling or optimization.
\item \emph{Compilation-based} methods, which perform inference by compiling
  the probabilistic program into an alternative representation which supports
  the kinds of queries that we care about.
\item \emph{Symbolic} methods, which operate directly on the program (i.e.,
  enumeration and block-based analyses).
\end{enumerate}

\subsection{Approximation Methods}

\subsubsection{Sampling}

\begin{enumerate}
\item \citep{Hur2015}: Utilizes program analysis to improve Markov-Chain Monte Carlo.
\item \citep{nori2014r2}: R2 sampling method
\item \citep{goodman2012church}: Church
\item \citep{wingate2011lightweight}: Sampling with program transformation
\item \citep{carpenter2017stan}: Stan
\end{enumerate}

\subsubsection{Variational Approximations}

\subsection{Compilation-based Methods}

\begin{enumerate}
\item \citep{Sampson2014}: Verifies that probabilistic assertions hold by
  compiling the program to a graphical model.
\item \citep{McCallum2009}: Factorie, a language for specifying factor graphs.
\item \citep{InferNET14}: Infer.NET, compiles probabilistic programs to factor
  graphs.
\item \citep{pfeffer2009figaro}: Figaro, compiles to factor graphs
\item \citep{pfeffer2001ibal}: Ibal, an early PPL which uses variable elimination
\item \citep{Fierens2013}: ProbLog, compiles probabilistic logic programs to
  weighted Boolean formulae.
\end{enumerate}

\subsection{Symbolic Methods}

\begin{enumerate}
\item \citep{Claret2013}: Performs inference by compiling probabilistic programs
  to algebraic decision diagrams (ADDs).
\item \citep{Sankaranarayanan:2013}: Approximates the probability with analyzing
  a finite subset of paths
\item \citep{Albarghouthi2017}: FairSquare, performs inference by approximating
  integrating under each path in the program.
\item \citep{BelleIJCAI15}: Approximate weighted model integration, a generalization of
  SMT-solvers to perform integrals instead of just finding a satisfying assignment.
\item \citep{Chistikov2015}: Performs inference using weighted model integration.
\end{enumerate}

\section{Program Transformations}
In standard program analysis, a program transformation is a rewriting procedure
which preserves the underlying semantics of the program; for example, the
optimization phase of a compiler. In the context of probabilistic programs, the
goal is to generalize well-known rewriting procedures to apply to programs with
probabilistic semantics, in the hopes of easing analyses such as inference.

\begin{enumerate}
\item \citep{Hur2014}: Generalizes program slicing to the setting of
  probabilistic programs with observations.
\item \citep{McIver2005}: Abstraction and refinement in probabilistic systems;
  extending the framework of program abstraction to verifying probabilistic
  properties.
\item \citep{wang2018pmaf}: PMAF, abstract interpretation for lower/upper bounds
  on Bayesian inference
\item \citep{narayanan2016probabilistic}: Hakaru, compiles programs into
  posterior distributions
\end{enumerate}


\section{Probabilistic Model Checking}

\begin{enumerate}
\item \citep{Baier1997}: Early work on symbolic model checking for probabilistic
  systems.
\item \citep{Hermanns2008}: Probabilistic CEGAR, generalizes CEGAR to
  probabilistic systems.
\item \citep{Dehnert2017ASI}: Storm model checker
\item \citep{kwiatkowska2002prism}: PRISM model checker
\end{enumerate}

\section{Applications}
\begin{itemize}
\item \citep{Foster2016}: Probabilistic network verification (ProbNetKat).
\item \citep{Gordon2014}: Using probabilistic programs to define probabilistic
  databases.
\item \citep{Schkufza2013}: Stochastic super-optimization; treats optimization as
  a search through a probability space over programs.

\end{itemize}


\bibliographystyle{plainnat}
\bibliography{bib}
\end{document}